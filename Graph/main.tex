
\section{Graph 图论}

\subsection{特殊图性质}

\paragraph{竞赛图:} 基图为无向完全图的有向简单图。

\begin{compactenum}
	\item 竞赛图强连通缩点后的DAG呈链状, 前面的所有点向后面的所有点连边。
	\item 任意竞赛图都有哈密顿路径;存在哈密顿回路当且仅当强联通。
  \item 竞赛图中大小为 $n$ 的强联通子图中存在大小为 $[3, n]$ 的环。
  \item 兰道定理(Landau's Theoerm): 
  不降序列$\{s_n\}$是合法的比分序列(即竞赛图的出度序列)当且仅当
  \(\forall 1 \leqslant k \leqslant n, \sum\limits_{i = 1}^{k}s_i \leqslant \tbinom{k}{2}\),
  且$\sum\limits_{i = 1}^n s_i = \tbinom{n}{2}$。
\end{compactenum}

\subsection{SP 最短路 Shortest Path}

Johnson 全源最短路 —— 任意图,复杂度 $\BigO(N^2 + N M \log M)$.

\cppinputlisting{Johnson}

\subsection{MST 最小生成树 Minimal Spanning Tree}

\subsubsection{矩阵树定理 Kirchhoff's matrix tree theorem}

\noindent the Laplacian matrix $L$ $=$ the degree matrix $D$ $-$ the adjacency matrix $A$. the number of spanning trees $=$ the absolute value of any cofactor of $L$.

\subsubsection{Kruskal(可判定唯一性)}

\cppinputlisting{MST/Kruskal}

\subsubsection{Prim}

\cppinputlisting{MST/Prim}

\subsubsection{Boruvka}

须保证:对于每个连通块,都能够找到与之距离最小的另一联通块。记对于一轮这一过程复杂度为 $O(p)$,那么最终的复杂度为 $O(p \log n)$. 

\subsubsection{XOR-MST 最小异或生成树}

可借助 字典树 用 Boruvka 算法求解,复杂度 $\BigO(n \log n \log a_i)$.

也可以对 字典树 dfs求解,复杂度 $\BigO(n \log \max(n, a_i))$.

\cppinputlisting{MST/xormst}

\subsection{网络流 Net Flow}

\cppinputlisting{NetFlow/Dinic}

\subsection{XX连通分量 XX-Connected Component}

\cppinputlisting{Tarjan-E-BCC}

\cppinputlisting{Tarjan-V-BCC}

\cppinputlisting{kosaraju}

\subsection{树の剖分 Decomposition}

\cppinputlisting{Heavy-Light-Decomposition}

\subsection{支配树 Dominator Tree}

\cppinputlisting{Dominator}
