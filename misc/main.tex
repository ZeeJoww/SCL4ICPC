
\section{Misc 杂项}

\subsection{C/C++ IO Cheat Sheet}

\subsubsection{read()}

\begin{cpplist}
#define getchar() \
    (tt==ss&&(tt=(ss=In)+fread(In,1,1<<20,stdin),ss==tt)?EOF:*ss++)
char In[1 << 20], *ss=In, *tt=In;
int read() {
    int x=0,f=1;
    char c=getchar();
    while(c<'0'||c>'9'){if(c=='-') f=-1;c=getchar();}
    while(c>='0'&&c<='9') x=x*10+c-'0',c=getchar();
    return x*f;
}
\end{cpplist}

\subsubsection{std::cin}

\begin{cpplist}
    // ref: https://blog.csdn.net/lingfeng2019/article/details/78463012
    // set base = 16, 8, 10 (reading integer) 
    std::cin >> (std::hex, std::oct, std::dec) >> number;
    // ignore next `count` chars (counting trailing '\0')
    istream & std::istream::ignore(int count = 1, int delim = EOF);
    // read next `count` chars to `buf` (counting trailing '\0')
    istream & std::istream::get(char * buf, int count, char delim = '\n');
    // read next `count` chars and don't add '\0' at the end of `buff`
    std::cin.read(buf, 5).read(buf + 5, 5);
    // peek (and don't read in) the next char
    char ch = std::cin.peek();
    // then we have: $peek() = get(ch) + putback(ch)$.
\end{cpplist}

\subsection{Python Helper}

记忆化装饰器 ~ \lstinline|@lru_cache(maxsize=128, typed=False)|

\begin{py3list}
@lru_cache(maxsize = None)  # None表示无限缓存
def fib(n):
    if n < 2:
        return n
    return fib(n - 1) + fib(n - 2)
\end{py3list}

\subsection{Bit Tricks}

\noindent \textbf{\textbf{References:}}
\begin{compactenum}[(i)]
\item \url{https://codeforces.com/blog/entry/98332}
\item \url{https://graphics.stanford.edu/~seander/bithacks.html}
\end{compactenum}

\begin{compactenum}[1.]

\item 异号判定

\begin{cpplist}
// Detect if two integers have opposite signs:
bool f = ((x ^ y) < 0); 
\end{cpplist}

\item 从高位加到低位的加法(见DFT/NTT)

\item 枚举子集、枚举超集

\begin{cpplist}
for (int s = t; s; s = (s - 1) & m) {} // subset
for (int s = t; s < max_state; s = (s + 1) | t) {} // superset
\end{cpplist}

\item 枚举所有掩码的子掩码复杂度 $\BigO(\displaystyle\sum_{m=1}^{2^n}2^m)=\BigO(3^n)$

\item 
next permutation

\begin{cpplist}
int t = x | (x - 1);
x = (t + 1) | (((~t & -~t) - 1) >> (__builtin_ctz(x) + 1));
\end{cpplist}

\end{compactenum}

\subsection{.vimrc}

\begin{lstlisting}[language=,
lineskip=-1.5pt,
frame=lines,
numbers=left,
numberstyle=\color{numberColor}\footnotesize\consolff,
basicstyle=\consolff\zihao{-5},
morecomment={[f][\color{gray}][0]{"}},
morecomment={[f][\color{gray}][12]{"}},
keywordstyle={\bfseries},
keywords={[2]{endfunc, set, color, nmap, map, func, exec, endfunc, set}}
keywordstyle={[2]{\bfseries}},
keywords={[3]{nu, ci, hls, ar, aw, is, et, sm, ai, ic, cul, cuc, nocp, noeb, smarttab, ts, hi, ch, so, bs, ls, sw, sts, mouse, completeopt, statusline}},
keywordstyle={[3]{\bfseries\color{blue}\consolff}},
showstringspaces=false]
set nu      "number
set ci      "cindent
set hls     "hlsearch
set ar      "autoread
set aw      "autowrite
set is      "incsearch
set et      "expandtab
set sm      "showmatch
set ai      "autoindent
set ic      "ignorecase
set cul     "cursorline
set cuc     "cursorcolumn
set nocp    "nocompatible
set noeb    "noerrorbells
set smarttab 

set ts=4    "tabstop
set hi=1000 "history
set ch=2    "cmdheight
set so=3    "scrolloff
set bs=2    "backspace
set ls=2    "laststatus
set sw=4    "shiftwidth
set sts=4   "sosttabstop
set mouse=a
set completeopt=longest,menu
set statusline=%F%m%r%h%w\ [TYPE=%Y]\ [POS=%l,%v]\ %{strftime(\"%H:%M\")}

color ron
" color torte
nmap tt :%s/\t/  /g<CR>

map <F8> :call Rungdb()<CR>
func! Rungdb()
exec "w"
exec "!g++ % -g -o gdb_%< && gdb ./gdb_%<"
endfunc
\end{lstlisting}

\subsection{Check list}

\begin{compactenum}
\item 数组是否需要排序?
\item 数据范围是否符合预期?
\item 模数是否正确?
\item 是否混用c/c++ IO?
\item 多组或多轮情形下: 初始化好了吗?
\item 下标起始是0还是1?是否与输入同步(加减1)?
\item debug 时修改的细节是否恢复?
\item 被左移的数是否须为 long long?
\item 输出格式是否匹配精度要求?精度过高是否会导致死循环?
\end{compactenum}
