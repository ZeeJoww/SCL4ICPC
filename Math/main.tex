
\section{Math 数学}

\subsection{Prime 素性}

\subsubsection{大数的判素与分解}

\cppinputlisting{Miller-Rabin}

\cppinputlisting{Pollard-Rho}

\pyinputlisting{prime}

\subsubsection{筛法}

能够$O(1)$计算素数$p$处值的积性函数均可使用\textbf{欧拉筛}在$O(n)$内预处理。

若$f(x)$和$g(x)$均为积性函数,则以下函数也是积性函数:
$$
f(x^p), 
\quad f^p(x), 
\quad f(x)g(x), 
\quad \displaystyle \sum_{d | x} 
      f(d) g\left(\displaystyle \frac{x}{d}\right)
$$

%todo : to add definition and formula 

欧拉函数前缀和(大数)

\cppinputlisting{phi-presum-1e10}

\subsubsection{莫比乌斯反演 mobius inversion}

useful conclusion: $n = \sum\limits_{d | n} \varphi(d)$

example: $\sum_{i = a}^{b} \sum_{j = c}^{d}[\gcd (i,j) = k]$

\cppinputlisting{Mobius}

\subsection{同余}

\subsubsection{exgcd}

\cppinputlisting{exgcd}

\cppinputlisting{another-exgcd}

\subsubsection{exCRT}

\cppinputlisting{exCRT}

\subsubsection{离散对数 (ex)bsgs}

\cppinputlisting{bsgs-basic}

\cppinputlisting{exbsgs}

\subsection{多项式 Polynomial}

\subsubsection{FFT/NTT}

\cppinputlisting{FFT}

\cppinputlisting{NTT}

\subsection{组合数学 Conbinatorics}

\cppinputlisting{fibonacci}

\subsubsection{Polynomial (MOD 998244353)}

\cppinputlisting{Poly}

\subsection{MAGIC PRIMES}
\begin{spacing}{0.9}
\begin{tabular}[ht!]{cccc}
\toprule
$r~2^k+1$&$r$&$k$&$g$\\
\midrule
3&1&1&2\\
5&1&2&2\\
17&1&4&3\\
97&3&5&5\\
193&3&6&5\\
257&1&8&3\\
7681&15&9&17\\
12289&3&12&11\\
40961&5&13&3\\
65537&1&16&3\\
786433&3&18&10\\
5767169&11&19&3\\
7340033&7&20&3\\
23068673&11&21&3\\
104857601&25&22&3\\
167772161&5&25&3\\
469762049&7&26&3\\
998244353&119&23&3\\
1004535809&479&21&3\\
2013265921&15&27&31\\
2281701377&17&27&3\\
3221225473&3&30&5\\
75161927681&35&31&3\\
77309411329&9&33&7\\
206158430209&3&36&22\\
2061584302081&15&37&7\\
2748779069441&5&39&3\\
6597069766657&3&41&5\\
39582418599937&9&42&5\\
79164837199873&9&43&5\\
263882790666241&15&44&7\\
1231453023109121&35&45&3\\
1337006139375617&19&46&3\\
3799912185593857&27&47&5\\
4222124650659841&15&48&19\\
7881299347898369&7&50&6\\
31525197391593473&7&52&3\\
180143985094819841&5&55&6\\
1945555039024054273&27&56&5\\
4179340454199820289&29&57&3\\
\bottomrule
\end{tabular}
\end{spacing}

\subsection{Notes}

\begin{compactenum}[I.]
\item 错排公式:$D(n) = (n-1)\left( D(n-1)+D(n-2) \right)$
\item 牛顿迭代(poly):
\(
g\left(f(x)\right) \equiv 0 \pmod{x^n} 
\Rightarrow 
f(x) \equiv f_0(x) - \frac{g\left(f_0(x)\right)}{g'\left(f_0(x)\right)}
\pmod{x^n}‘
\)
\item 卡特兰数与路径计数
  \begin{compactenum}[1.]
    \item 第 $n$ 个卡特兰数 $H_n = \frac{1}{n + 1}\binom{2n}{n} = \frac{H_{n - 1}(4n - 2)}{n + 1}$.
    \item 经典递推式 $H_n = \begin{cases}
        \sum\limits_{i = 1}^{n} H_{i - 1} H_{n - i} & n \geqslant 2 \\
        1 & n \in \{0, 1\}
    \end{cases}$
    \item 圆上 $2n$ 个点成对连接($n$ 个匹配)不相交的方案数为 $H_n$.
    \item 从 $(0, 0)$ 到 $(n, n)$ 的不穿过直线 $y = x$ 的非降路径数为 $\frac{2}{n + 1}\binom{2n}{n}$.
    \item 从 $(0, 0)$ 到 $(n, n)$ 的除起点与终点外均不接触直线 $y = x$ 的非降路径数为 $2\binom{2n - 2}{n - 1} - 2\binom{2n - 2}{n}$.
    \item 从 $(0, 0)$ 到 $(n, n)$ 的不穿过直线 $y = x - m$ ($m \geqslant 1$) 的非降路径数为 $\binom{2n}{n} - \binom{2n}{n - m - 1}$.
    \item 从 $(0, 0)$ 到 $(n, m)$ ($n > m$)的不穿过直线 $y = x$ 的非降路径数为 $\binom{n + m}{n} - \binom{n + m}{n + 1}$.
    \item 生成函数 $H(x) = \sum\limits_{k = 1}^{\infty} A_k x^k = \frac{1 - \sqrt{1 - 4x}}{2x}$.
  \end{compactenum}
\end{compactenum}
