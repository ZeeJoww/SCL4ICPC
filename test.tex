\section{TEST}

\subsection{ALL todo lists}

\subsubsection{Math}

\begin{todolist}

\item 素性:杜教筛,Min25-筛

\item 数论函数:狄利克雷卷积,莫比乌斯反演

\item 线性代数:线性基,常系数线性递推

\item 多项式:拉格朗日插值,集合幂级数(FWT/FMT)

\item 组合数学:错排,卡特兰数,斯特林数,伯努利数,BM(最短线性递推),min-max容斥,二项式反演,prufer序列

\item 群论:置换,Burnside定理,Polya定理

\item 数值积分:辛普森,自适应辛普森

\end{todolist}

\subsubsection{String}

\begin{todolist}

\item (?):后缀自动机,回文自动机,最小表示法,Lyndon分解

\end{todolist}

\subsubsection{Data-Structure}

\begin{todolist}

\item 堆:对顶堆

\item 区间操作:树状数组求第k大,二维树状数组,李超线段树,线段树合并;

\item 树相关:替罪羊树;笛卡尔树,虚树,kd-tree,析合树;长链剖分

\item 并查集:带权并查集,可持续化并查集

\item 分块:莫队,链上分块,树上分块

\item misc:DLX(Dancing Links)

\end{todolist}

\subsubsection{Graph}

\begin{todolist}

\item 最短路:差分约束,k短路

\item 连通分量:圆方树

\item 二分图:匈牙利,KM,Hopcraft-Karp

\item 网络流:SAP;最大流,可行流,zkw费用流(?);上下界网络流

\item misc:欧拉回路,2-SAT,斯坦纳树,3/4元环,最小树形图,一般图匹配,最小瓶颈路,全局最小割

\end{todolist}

\subsubsection{CG}

\begin{todolist}

\item 工具:交点,Voronoi,最小圆覆盖

\end{todolist}

%\begin{center}
%\begin{tikzpicture}
%\begin{axis}[x={(10,0)},y={(0,10)},z={({-10/sqrt(8)},{-10/sqrt(8)})},
%minor tick num=4,xmin=0,xmax=9.9,ymin=0,ymax=9.9,zmin=0,
%zmax=12.7,ticks=none,clip=false,tick style={opacity=0}]
%\draw[->, >=latex, ultra thick] (0,0,-2.6) -- (0,0,13) node[left]{$x_1$};
%\draw[->, >=latex, ultra thick] (0,0,0) -- (10,0,0) node[below]{$x_2$};
%\draw[->, >=latex, ultra thick] (0,0,0) -- (0,10,0) node[left]{$x_3$};
%\draw[fill=green!50!white, nearly transparent] (0,0,0) -- (0,0,12.7) -- (0,9.9,12.7) -- (0,9.9,0);
%\draw[fill=blue!50!white, nearly transparent] (0,0,0) -- (9.9,0,0) -- (9.9,9.9,0) -- (0,9.9,0);
%\draw[fill=red!50!white, nearly transparent] (0,0,0) -- (9.9,0,0) -- (9.9,0,12.7) -- (0,0,12.7);
%
%\addplot3[->,no marks,blue,ultra thick] coordinates {(4,0,5.2) (4,4,5.2)};
%\addplot3[->,no marks,red,ultra thick] coordinates {(0,0,5.2) (4,0,5.2)};
%\addplot3[->,no marks,green,ultra thick] coordinates {(0,0,0) (0,0,5.2)};
%\shade[ball color=yellow] (4,4,5.2) circle (0.2);
%\end{axis}
%\end{tikzpicture}
%
%
%\begin{figure}[htb]
%\centering
%\begin{tikzpicture}[>=Stealth] %修改箭头样式
%  \draw[thick] (1,0) .. controls (2,-1) and (3.2,-1.2) 
%     .. (4,0) .. controls (5,2) and (5,3) 
%     .. (4,4) .. controls (3,5) and (2,5)
%     .. (0,4) .. controls (-1,3.5) and (-1,2.1)
%     .. (0,1) -- (1,0); % 若尔当曲线
%  \draw[->] (-1,0) -- (5.5,0) node[below] {$x$}; 
%  \draw[->] (0,-1) -- (0,5) node[left] {$y$}; 
%  \node[below left] at (0,0) {$O$}; 
%  \node[below right] at (3.5,-0.5) {$c$};
%  \node at (2,2) {$I(c)$};
%  \node[right] at (5,3) {$E(c)$};
%\end{tikzpicture}
%\caption{若尔当定理示意图}\label{fig:RCT}
%\end{figure}
%\end{center}



%\setcounter{page}{55}
%\newpage
%
%\section*{Hot-Fix 补丁}
%
%\setcurdir{Math}
%
%\cppinputlisting{phi-presum-1e10}
%
%\setcurdir{String}
%
%\cppinputlisting{PAM}
%
%Lyndon 串:字典序严格小于自身所有非平凡后缀的字符串。
%
%Lyndon 分解: $s = w_1 + \cdots + w_k$,其中 $w_i$ 是 Lyndon 串且 $w_1 \geqslant \cdots \geqslant w_k$.
%
%\cppinputlisting{Lyndon}
%
%\cppinputlisting{minimal-cyclic-shift}
%
%\cppinputlisting{extsam}
%
%\setcurdir{Graph}
%
%\cppinputlisting{MST/xormst}